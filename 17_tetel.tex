\documentclass[]{article}
\usepackage{lmodern}
\usepackage{amssymb}
\usepackage{amsmath}
\usepackage{polyglossia}
\usepackage{listings}
\usepackage{tcolorbox}
\usepackage{etoolbox}
\usepackage{setspace}
\usepackage{framed}
\usepackage{hyperref}
\usepackage[a4paper,margin=2cm,footskip=.5cm]{geometry}
\newcommand{\R}{\mathbb{R}}
\newcommand{\Rn}[1]{$\mathbb{R}^{#1}$}
\newcommand{\Und}[1]{\underline{#1}}
\definecolor{shadecolor}{gray}{0.9}
%opening 
\title{Bevezetés a Számításelméletbe 2.\\{\large 13. tétel}}
\author{Hegyi Zsolt}
\begin{document}
\maketitle
\url{http://cs.bme.hu/bsz2/dfs.pdf}
\begin{framed}
A DFS algoritmus a következő adatokat tartja számon:
\begin{itemize}
\item d(v): a v csúcs mélységi száma
\item f(v): a v csúcs befejezési száma
\item m(v): a v-t megelőző csúcs - tehát amiből a v-t a bejárás elérte
\item a: a jelenleg aktív csúcs
\item g: az aktuális gyökérpont
\item D: az eddigi legnagyobb mélységi szám
\item F: az eddigi legnagyobb befejezési szám
\end{itemize}
\end{framed}
\begin{framed}
DFS Algoritmus: Bemenet: Egy n csúcsú G irányított gráf és egy $s \in V$ csúcs.
\begin{itemize}
\item{\textbf{0.}} $d(s) = 1$, minden $v \neq s$-re $d(v) = *$, minden v-re $v = *$, minden v-re $m(v) = *$, $a = s$, $g = s$, $D = 1$, $F = 0$.
\item{\textbf{1.}}
\\
HA létezik olyan $e = \overrightarrow{av}$ él, melyre $d(v) = *$, AKKOR:
	\begin{itemize}
	\item $D = D + 1$
	\item $d(v) = D$
	\item $m(v) = a$
	\item $a = v$
	\item \textbf{1.} lépéshez vissza
	\end{itemize}
\item{\textbf{2.}}
	\begin{itemize}
	\item $F = F + 1$
	\item $f(a) = F$
	\item HA $a \neq g$, AKKOR $a = m(a)$ és \textbf{1.} lépéshez vissza
	\item HA $D = n$, AKKOR \textbf{STOP}.
	\item Válasszunk olyan v csúcsot, melyre $d(v) = *$.
	\item $g = v$, $a = v$, \textbf{1.} lépéshez vissza.
	\end{itemize}
\end{itemize}
\end{framed}
\begin{shaded}
DFS-ERDŐ Definíció: s csúcsból indítva G irányított gráfban lefuttatuk a DFS algoritmust. A futáshoz tartozó DFS erdő F. Legyen $e=\overrightarrow{uv}$ a G-nek tetszőleges éle. Ekkor
\begin{itemize}
\item e-t faélnek nevezzük, ha $e \in E(F)$.
\item e-t előreélnek nevezzük, ha nem faél, de F-ben van u-ból v-be irányított út (vagyis v leszármazottja u-nak).
\item e-t visszaélnek nevezzük, ha F-ben van v-ből u-ba irányított út (tehát v őse u-nak).
\item e-t keresztélnek nevezzük, ha F-ben sem u-ból v-be, se v-ből u-ba nincs irányított út.
\end{itemize}
\end{shaded}
\begin{framed}
DFS ÉS ACIKLIKUSSÁG Tétel: Ha G irányított gráf aciklikus, a DFS futtatásakor nem keletkezik visszél. Ha nincs visszél, akkor f(v) szerinti csökkenő sorrend a topologikus sorrend.
\end{framed}
\begin{leftbar}
Bizonyítás: Ha keletkezik visszél és $e = \overrightarrow{uv}$ ilyen, akkor G irányított gráf nyilván tartalmaz irányított kört: az F DFS-erdp tartalmaz v-ből u-ba irányított utat (mert e visszél), amit e-vel kiegészítve irányított körré zárhatunk. Tegyük fel ezért, hogy nem keletkezik visszaél. Ha megmutatjuk a tétel 2. állítását, hogy a csúcsoknak a befejezési számozás szerinti fordított sorrendje topologikus rendezés, akkor ebből nyilván következik G aciklikussága is. Azt kell megmutatnunk, hogy G minden $e = \overrightarrow{uv}$ élére $f(v) < f(u)$. Mivel feltettük, hogy visszaél nem keletkezett, ezért e lehet faél, keresztél vagy előreél. Ha e faél vagy előreél, akkor a DFS eljárás során u első aktívvá válásakor az u befejezéséig tartó szakasza során érte el v-t, így ennek során kellett befejeznie azt, tehát a v-t előbb fejezte be u-nál, tehát $f(v) < f(u)$. Ha pedig e keresztél, akkor e vizsgálatának a pillanatában f(v) már kapott értéket, u viszont épp aktív csúcs volt, így ekkor még $f(u) = *$ volt. Mivel az eljárás egyre nagyobb befejezési számokat ad, ezért $f(v) < f(u)$ erre is teljesülni fog.
\end{leftbar}
\end{document}