\documentclass[]{article}
\usepackage{lmodern}
\usepackage{amssymb}
\usepackage{amsmath}
\usepackage{polyglossia}
\usepackage{listings}
\usepackage{tcolorbox}
\usepackage{etoolbox}
\usepackage{setspace}
\usepackage{framed}
\usepackage[a4paper,margin=2cm,footskip=.5cm]{geometry}
\newcommand{\R}{\mathbb{R}}
\newcommand{\Rn}[1]{$\mathbb{R}^{#1}$}
\newcommand{\Und}[1]{\underline{#1}}
\definecolor{shadecolor}{gray}{0.9}
%opening 
\title{Bevezetés a Számításelméletbe 2.\\{\large 8. tétel}}
\author{Hegyi Zsolt}
\begin{document}
\maketitle
\begin{framed}
MOHÓ SZÍNEZÉS Algoritmus: A mohó színezés sorbarendezi a csúcsokat ($v_1, v_2...v_n$) és a $v_i.$-hoz azon legkisebb színt rendeli, amit a ($v_1,...v_{i-1}$) szomszédokhoz még nem rendelt. A mohó színezés nem feltétlenül a legoptimálisabb színezést adja.
\end{framed}
\begin{framed}
$\chi(G)$ ÉS $\Delta(G)$ VISZONYA Tétel: Minden	G gráfra teljesül, hogy
$$\chi(G) \leq \Delta(G)+1$$
\end{framed}
\begin{leftbar}
Bizonyítás: A mohó színezés segítségével bizonyítjuk. Színezzük ki G pontjait ($v_1, v_2,..., v_n$) úgy, hogy az i-edik lépésben $v_i$-t olyan színre színezzük, ami nem szerepel $v_i$ kiszínezett szomszédságában. Mivel $v_i$-nek legfeljebb $\Delta(G)$ kiszínezett szomszédja lehet, és mindegyik szomszéd legfeljebb egy-egy színt zár ki, ezért $v_i$ színezése elvégezhető a rendelkezésre álló színek valamelyikével (a kimaradó $\Delta(G)+1)$-edik). $v_n$ kiszínezése után G egy $(\Delta(G)+1)$ színezését kapjuk meg.
\end{leftbar}
Érdemes megjegyezni, hogy $\chi(G) = \Delta(G)+1$ abban az esetben, ha teljes gráfról vagy húr nélküli páratlan körről beszélünk.
\begin{framed}
BROOKS-TÉTEL Tétel: Ha G egyszerű, összefüggő, nem teljes gráf, nem páratlan hosszúságú kör, akkor $\chi(G) \leq \Delta(G)$, tehát a kromatikus szám nem nagyobb, mint a maximális fokszám.
\end{framed}
\begin{shaded}
INTERVALLUMGRÁF Definíció: Legyenek $I_1 = [a_1, b_1], I_2 = [a_2, b_2],...$ korlátos zárt intervallumok, és minden $a_i, b_i$ legyen pozitív egész. Legyenek $p_1, p_2,...$ egy G gráf pontjai és ${p_i, p_j}$ akkor és csak akkor legyen él G-ben, ha $I_i\cap I_j \not= \emptyset$. Az így előálló gráfokat \textbf{intervallumgráfnak} nevezzük.
\end{shaded}
\begin{framed}
INTERVALLUMGRÁF SZÍNEZÉSE Tétel: G intervallumgráf, emiatt:
$$\omega(G) = \chi(G)$$
\end{framed}
\begin{leftbar}
Mohó színezéssel bizonyítjuk, méghozzá úgy, hogy az intervallumokat bal végpontja szerint rendezem és ezek alapján növekvő sorrendben színezem, akkor optimális színezést kapok. T.f.h. k színnel színez a mohó algoritmus. A cél az, hogy belássuk a következőt:
$$k \leq \omega(G) \leq \chi(G) \leq k$$. MAJD.
\end{leftbar}
\begin{shaded}
PÁROS GRÁF Definíció: Egy G gráfot \textbf{páros gráfnak} nevezünk, ha G pontjainak V(G) halmazát két részre, egy A és B halmazra tudjuk osztani úgy, hogy G minden élének egyik végpontja A-ban, a másik pedig B-ben van. Ennek jelölése: $G = (A,B)$. A $K_{a,b}$-vel jelölt teljes páros gráf olyan $G=(A,B)$ gráf, ahol $|A| = a$ és $|B| = b$ és minden A-beli pont össze van kötve minden B-beli ponttal.
\end{shaded}
\begin{framed}
PÁROSÍTÁS LÉTEZÉSE Tétel: Egy G gráf akkor és csak akkor páros gráf, ha minden G-ben lévő kör páros.
\end{framed}
\begin{leftbar}
Bizonyítás: Ha G páros gráf, és C egy kör G-ben, akkor C pontjai felváltva vannak A-ban és B-ben, így $|V(C)|$ nyílván páros. Ha G minden köre páros hosszú, akkor megadhatjuk az A és B halmazt. Válasszunk egy tetszőleges v pontot, legyen ez A első pontja.  v minden szomszédját tegyük bele B-be, majd ezeknek a szomszédjait rakjuk bele A-ba. Ezt folytassuk, amíg ki nem fogyunk a pontokból. Ez biztosan jó elosztás, mivel ha például lenne A-ban két szomszédos pont, akkor léteznie kéne a gráfban páratlan körnek, így ellentmondásra jutnánk. Nem öf. gráfok esetén komponensenként hajtsuk végre.
\end{leftbar}
\begin{framed}
KROMATIKUS SZÁM ÉS PÁROSÍTÁS KAPCSOLATA Tétel: Egy legalább egy élet tartalmazó G gráf akkor és csak akkor páros, ha $\chi(G) = 2$.
\end{framed}
\begin{leftbar}
Ha a gráf páros, akkor az egyik oldalon lévő pontokat pirossal, másik oldal lévőeket megszínezzük kékkel. Ha a gráfnak van egy éle, akkor ennek a két végpontját már nem színezhetjük egy színűre. A színek megfelelnek a két halmaznak, amire fel tudjuk bontani a páros gráfokat.
\end{leftbar}
\end{document}
