%%%% Beállítások, importok

%% BME-Notes jegyzethez header
%% Ha nem érted, mi történik itt, akkor inkább ne változtasd meg!
%% A fájlok fordításához XeLaTeX-et kell használni
\documentclass[]{article}
\usepackage{lmodern}
\usepackage{amssymb}
\usepackage{amsmath}
\usepackage{polyglossia}
\usepackage{listings}
\usepackage{tcolorbox}
\usepackage{etoolbox}
\usepackage{setspace}
\usepackage{framed}
\usepackage[a4paper,margin=2.5cm]{geometry}
\usepackage{fancyhdr}
\pagestyle{fancy}
\usepackage[hidelinks]{hyperref}
\definecolor{shadecolor}{HTML}{eeeeee} %Kindle-optimized color
\setcounter{secnumdepth}{0} %this automagically removes extra numbering toc and sections
\renewcommand{\contentsname}{Tartalomjegyzék}
\newtoks\cim
\newtoks\szerzo
\newtoks\segitettek
\newtoks\datum

\newcommand{\ujfejezet}[1]{\newpage \input{./fejezetek/#1.tex}}
\newenvironment{tetel}[1]{\begin{framed}\noindent\ignorespaces\textbf{\large Tétel: #1}\normalsize\\}{\end{framed}\ignorespacesafterend}
\newenvironment{definicio}[1]{\begin{shaded}\noindent\ignorespaces\textbf{\large Definíció: #1}\normalsize\\}{\end{shaded}\ignorespacesafterend}
\newenvironment{bizonyitas}[1]{\begin{leftbar}\noindent\ignorespaces\textbf{\large Bizonyítás: #1}\normalsize\\}{\end{leftbar}\ignorespacesafterend}


\title{\huge\textsc{\the\cim}}
\author{\the\szerzo}
\date{\the\datum}


\newcommand{\R}{\mathbb{R}}
\newcommand{\Rn}[1]{$\mathbb{R}^{#1}$}
\newcommand{\Und}[1]{\underline{#1}}

%%%%
%%%%%%%
%%%% Ezeket változtasd meg!

\cim{Bevezetés a Számításelméletbe 2.}
\datum{2017. január 13.}
\szerzo{Hegyi Zsolt}
\segitettek{Bálint Ádám, Bereczki Márk, Bognár Márton, Bokros Bálint, Braun Márton, a CrySyS labor munkatársai,  Hanusch Róbert, az IRC-s baráti társaságom, Kormány Zsolt, a KSZK reszort tagjai, Müller András, Nagy "Sid" Jenő, Rostás Balázs.}

%%%%
%%%%%%%
%%%% Fedlap

\begin{document}
\begin{titlepage}
		\centering
		\vspace{5cm}\par
		\maketitle
		\large A jegyzet és annak forrása megtalálható a \texttt{bme-notes.github.io} weboldalon.
		\vfill
		Közreműködtek: \the\segitettek
		\normalsize
		% Bottom of the page
\end{titlepage}

%%%%
%%%%%%%
%%%% Tartalomjegyzés + előszó

\noindent \textbf{Kellemes vizsgázást!}

\tableofcontents{}

\section{Előszó}
A tételsor Fleiner Tamás jegyzetei, a "Számítástudomány alapjai" című könyv (Katona Gyula Y., Recski András), valamint Szeszlér Dávid fantasztikus előadásai alapján készült.

%%%%
%%%%%%%
%%%% Tényleges content
%%%% A fejezeteket a fejezetek/$NEV.tex elérési útvonalon keresi
%% Első "fejezet"
\ujfejezet{01_tetel}
\ujfejezet{02_tetel}
\ujfejezet{03_tetel}
\ujfejezet{04_tetel}
\ujfejezet{05_tetel}
\ujfejezet{06_tetel}
\ujfejezet{07_tetel}
\ujfejezet{08_tetel}
\ujfejezet{09_tetel}
\ujfejezet{10_tetel}
\ujfejezet{11_tetel}
\ujfejezet{12_tetel}
\ujfejezet{13_tetel}
\ujfejezet{14_tetel}
\ujfejezet{15_tetel}
\ujfejezet{16_tetel}
\ujfejezet{17_tetel}


\end{document}
