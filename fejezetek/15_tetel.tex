\section{12. tétel: Legrövidebb utak adott csúcsból: Dijsktra és Ford algoritmusai}

% \begin{tetel}{Az algoritmusok futása során a következőket tartjuk számon}
Az algoritmusok futása során a következőket tartjuk számon:

\begin{itemize}
\item[] l(e) - az e él hossza
\item[] d(v) - a v pontba az s kezdőpontból eddigi legrövidebb út hossza. d(s) = 0.
\end{itemize}
Kulcslépések:
\begin{itemize}
\item[\textbf{(*)}]$d(s) = 0$, minden $v \neq s$-re $d(v) = \infty$
\item[\textbf{(**)}]Ha x-ből vezet egy e él y-ba és $d(y) > d(x) + l(e)$, akkor $d(y) = d(x) + l(e)$.
\end{itemize}
%\end{tetel}

\begin{tetel}{DIJKSTRA-ALGORITMUS}
\textbf{Az algoritmus:}
\begin{itemize}
\item[\textbf{0.}] $S = {s}, T = V\backslash {s}$ és \textbf{(*)}.
\item[\textbf{1.}] Minden S-beli pontból minden T-beli pontba vezető e élre végezzük el \textbf{(**)} javítást.
\item[\textbf{2.}] A T-beli pontok közül legyen $v_0$ az, amelyiken a d(v) érték a legkisebb. Tegyük át $v_0$-t T-ből S-be.
\item[\textbf{3.}] Ha T üres, \textbf{STOP}. Ha nem, vissza \textbf{1.} lépéshez.
\end{itemize}
Az algoritmus lépésszáma $c\cdot n^3$, mivel az 1. lépés k. elvégzésekor $|S| = k,\,\, |T| = v - k$, így az összes \textbf{(**)} hívások száma $\sum k(v-k) = \sum kv - \sum k^2$ és ennek az összege $k^3$-höz közelít.\\Az algoritmusnak létezik egy kedvezőbb futási idejű változata, $c\cdot n^2$ lépésszámmal.\\
\textbf{Az optimalizált algoritmus:}
\begin{itemize}
\item[\textbf{0.}] $S = {s}, T = V\backslash {s}$ és \textbf{(*)}, valamint $v_0 = s$.
\item[\textbf{1.}] Csak a $v_0$-ból a T-beli pontokba vezető e élekre végezzük el a \textbf{(**)} javítást.
\item[\textbf{2.}] A T-beli pontok közül legyen $v_0$ az, amelyiken a d(v) érték a legkisebb. Tegyük át $v_0$-t T-ből S-be.
\item[\textbf{3.}] Ha T üres, \textbf{STOP}. Ha nem, vissza \textbf{1.} lépéshez.
\end{itemize}
\end{tetel}

\begin{tetel}{FORD-ALGORITMUS}
A Ford-algoritmus megengedi a negatív súlyú éleket is, valamint az algoritmus egyszerűbb, mint a Dijsktra-algoritmus.
\begin{itemize}
\item[\textbf{0.}] Számozzuk meg az éleket 1-től e-ig, ezt rögzítsük le (tetszőleges sorrend). Legyen $i = 1$ és \textbf{(*)}.
\item[\textbf{1.}] A rögzített sorrendben végezzük el a \textbf{(**)} javítást minden élen.
\item[\textbf{2.}] $i = i + 1$. Ha $i > v$, akkor \textbf{STOP}. Különben folytassuk \textbf{1.} lépésnél.
\end{itemize}
Az algoritmus lépésszáma $c\cdot e\cdot v$, ez jóval nagyobb általában, mint $n^2$, ezt az árat kell megfizetnünk a negatív élhossz feature-ért. Mi történik negatív kör esetén? Ezt valahogyan fel kell ismerni! A módosított 2. lépés, ami jelzi, ha negatív összsúlyú körbe kerültünk:
\begin{itemize}
\item[\textbf{2.}] Ha az \textbf{1.} lépés során egyetlen javítás sem történt, akkor \textbf{STOP} (és megvannak a minimális úthosszak). Különben $i = i + 1$. Ha $i \leq v + 1$, folytassuk \textbf{1.} lépésnél, ha pedig $i > v + 1$, akkor \textbf{STOP} (és van negatív összsúlyú kör).
\end{itemize}
\end{tetel}
