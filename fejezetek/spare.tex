\section{Egyéb :)}

\begin{tetel}{VÁGÁSOK ÉS KÖRÖK}
G összefüggő és síkbarajzolt, ekkor ha C kör a G-ben, a C* vágás lesz G duálisában, G*-ban és fordítva, ha C vágás G-ben, a C* kör lesz a G*-ban.
\end{tetel}

\begin{tetel}{FA DUÁLISON BELÜL}
G összefüggő, egyszerű gráf, F ezen belül feszítőfa. A G* duálison belül ez az F a saját komplementereként fog megjelenni.
\end{tetel}

\begin{tetel}{DISZJUNKT FOLYAMOK}
Ha a kapacitások egész számok, akkor van olyan maximális folyam, melyben minden élen a folyam értéke egész. Így nyilvánvaló, hogy ha a kapacitás minden élen 1 vagy 0, akkor van olyan maximális folyam, melynek minden élén a folyam értéke vagy 1 vagy 0. Ha elhagyjuk ez utóbbi éleket, akkor diszjunkt utakat kapunk s-ből t-be. Ezeknek a számát úgy is meg tudjuk kapni, hogy veszünk egy minimális vágást és az élhalmazának az elemszámával lesz egyenlő a diszjunkt utak száma.
\end{tetel}

\begin{tetel}{DISZJUNKT FOLYAM ALGORITMUS}
Vegyünk tetszőleges hálózatot, és futtassuk le a fentebb leírt módszert rajta úgy, hogy vegyünk egy minimális vágást a hálózatban. A visszaélek (tehát amik a t-t tartalmazó halmazból az s-et tartalmazó halmazba mennek) legyenek 0 értékűek, egyébként pedig 1 értékűek az élek. Ebben már meg lehet keresni a diszjunkt utakat.
\end{tetel}