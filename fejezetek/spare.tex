\section{Egyéb :)}

\begin{definicio}{ELVÁGÓ ÉLHALMAZ}
G összefüggő gráf, $x \in E(G)$. x elvágó élhalmaz, ha $(V(G),E(G)\backslash x) = G'$ és G' nem összefüggő. x vágás, ha x elv. élhalmaz, de semelyik részhalmaza sem az.
\end{definicio}

\begin{tetel}{VÁGÁSOK ÉS KÖRÖK}
G összefüggő és síkbarajzolt, ekkor ha C kör a G-ben, a C* vágás lesz G duálisában, G*-ban és fordítva, ha C vágás G-ben, a C* kör lesz a G*-ban.
\end{tetel}

\begin{tetel}{FA DUÁLISON BELÜL}
G összefüggő, egyszerű gráf, F ezen belül feszítőfa. A G* duálison belül ez az F a saját komplementereként fog megjelenni.
\end{tetel}
