\section{11. tétel}

\begin{shaded}
REGULÁRIS GRÁF Definíció: Egy gráfot \textbf{$k$-regulárisnak} nevezünk, ha minden pont foka $k$. Egy gráfra azt mondjuk, hogy \textbf{reguláris}, ha létezik olyan k, amire k-reguláris.
\end{shaded}
\begin{framed}
REGULÁRIS GRÁFBAN TELJES PÁROSÍTÁS Definíció: Minden reguláris páros gráfban létezik teljes párosítás.
\end{framed}
\begin{leftbar}
Bizonyítás:
Vegyünk egy páros gráfot A, B pontosztállyal, ami k-reguláris. Először lássuk be, hogy ugyanaz az elemszáma A és B-nek. A-ból $k\cdot n$ él megy ki, B-be pedig $k \cdot m$ él megy. $k\cdot n = k\cdot m$, osszuk le k-val $\rightarrow n = m$. Hall-feltétel: $|N(A) \geq |A|$. A-ból $k\cdot|A|$ él megy ki, ezek a N(A) csúcsokba mennek. Ez annyit jelent, hogy egy N(A)-beli csúcsba átlagosan $(k\cdot |A|) \backslash |N(A)|$ él megy, tehát $(k\cdot |A|) \backslash |N(A)| \leq k$ (mivel egy csúcsba maximum k él megy). Szorozzuk be $|N(A|$-val és osszunk k-val. $|A| \leq |N(A)|$, Hall feltétel OK, Frobenius-tétel OK, létezik teljes párosítás.
%Tegyük fel, hogy G k-reguláris és rendelkezik A, B pontosztályokkal és $k \geq 1$, és tekintsünk tetszőleges $X \subseteq A$ ponthalmazt. Mivel X-beli pontokból összesen $|X|\cdot k$ darab él indul ki, és N(X) minden pontjába ezek közül legfeljebb k fut be, ezért
%$$|N(X)| \geq \frac{1}{k}k|X| = |X|$$
%Az $X = A$ esetre ugyanezzel a módszerrel $|A| = |B|$, mert $N(A) = B$ és így az A-beli pontokból kiinduló élekből N(A) minden pontjába pontosan k él fut be. Ebből Frobenius tétellel kijön a teljes párosítás.
\end{leftbar}
\begin{shaded}
ÉLSZÍNEZÉS Definíció: Egy G gráf élei \textbf{k színnel színezhetőek}, ha minden élt ki lehet színezni k színnel úgy, hogy bármely két szomszédos él színe különböző legyen. G \textbf{élkromatikus száma} $\chi_e(G) = k$, ha G élei k színnel színezhetőek, de $k - 1$-gyel már nem.
\end{shaded}
\begin{framed}
VIZING-TÉTEL Tétel: Ha G egyszerű gráf, akkor $\chi_e(G) \leq \Delta(G) + 1$. Bizonyítás $\emptyset$.
\end{framed}
\begin{framed}
ÉLKROMATIKUS SZÁM Tétel: Tetszőleges G gráfra $\chi_e(G) \geq \Delta(G)$ áll.
\end{framed}
\begin{leftbar}
Bizonyítás: Az egy csúcsból induló élek egymástól különböző színt kapnak, és ez speciálisan a maximális fokszámú csúcsokból induló élekre is igaz.
\end{leftbar}
\begin{framed}
KŐNIG-TÉTEL Tétel: Ha G páros gráf, akkor $\chi_e(G) = \Delta(G)$
\end{framed}
\begin{leftbar}
Bizonyítás: Elég azt igazolni, hogy $\chi_e(G) \leq \Delta(G)$ az előző állítás miatt. Létezik olyan H páros gráf, melynek G részgráfja, és H minden csúcsának fokszáma $\Delta(G)$. Ha sikerül a $\Delta(G)$-reguláris H gráf éleit $\Delta(G)$ színnel kiszínezni, akkor egyúttal a G részgráf éleinek is megkapjuk egy ugyanennyi színnel való színezését. H gráf élszínezéséhez elegendő azt megmutatni, hogy tetszőleges reguláris páros gráfban létezik teljes párosítás, ezt pedig már bizonyítottuk egy fentebbi tételben.
\end{leftbar}
