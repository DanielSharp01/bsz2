\section{3. tétel}

\url{http://cs.bme.hu/bsz2/bfs.pdf}

\begin{tetel}{A BFS-algoritmus futása során a következőket tartjuk nyilván}
\begin{itemize}
\item $b(i) (i = 1,2,3,...)$: az i-edikként bejárt csúcs
\item $t(v) (v \in V)$: v távolsága s-től
\item $m(v) (v \in V, v \neq s)$: v-t megelőző csúcs az algoritmus által megtalált, s-ből v-be vezető legrövidebb úton
\item j: az eddig bejárt csúcsok száma
\item k: a jelenleg aktív csúcs sorszáma a $b(1), b(2),...$ sorozatban.
\end{itemize}
\end{tetel}

\begin{tetel}{BFS Algoritmus}
Bemenete a BFS algoritmusnak: Egy G gráf és egy $s \in V$ csúcs.
\\
\textbf{Az algoritmus:}
\begin{itemize}
\item{\textbf{0.}} $j = 1$, $k = 1$, $b(1) = s$, $t(s) = 0$, minden $v \neq s$-re $t(v) = *$
\item{\textbf{1.}} HA a b(k) csúcsnak van olyan v szomszédja, amelyre $t(v) = *$, AKKOR:
	\begin{itemize}
	\item $j = j + 1$
	\item $b(j) = v$
	\item $t(v) = t(b(k)) + 1$
	\item $m(v) = b(k)$
	\item Vissza az \textbf{1.} lépéshez.
	\end{itemize}
\item{\textbf{2.}}
	\begin{itemize}
	\item Ha $k = j$, akkor \textbf{STOP}.
	\item $k = k + 1$
	\item Vissza az \textbf{1.} lépéshez.
	\end{itemize}
\end{itemize}

Mivel van, akinek a matematikus notáció nem túl olvasható (pl. ezen részlet szerzője), ezért alább olvasható ugyanezen algoritmus C++-szerűen refaktorálva, bizonyos Clean Code szempontokra ügyelve. Az alább látható algoritmus ténylegesen refaktorálás és teljesen egyenértékű a fentivel; nem tartalmaz módosításokat.

\afterpage{%
  \begin{framed}
    \noindent \large{\textbf{BFS algoritmus, C++-szerűen}}
      \lstinputlisting[language=C]{res/bfs.cpp}

  \end{framed}
  \clearpage
}


Az algoritmus lineáris futásidejű, tehát $c \cdot e$ lépésszámú.

\end{tetel}

\begin{definicio}{BFS-FA}
A BFS algoritmus futtatása után kapott F feszítőfát nevezzük \textbf{BFS fának}. F összefüggő, ha az eredeti bemeneti gráf is összefüggő volt, valamint F nem tartalmaz kört (a fa definíciója miatt). Az is megfigyelhető, hogy bármely $v \in V$ csúcsra az s-et v-vel összekötő F-beli út a legrövidebbek egyike az s-ből a v-be vezető G-beli utak közül.
\end{definicio}

\begin{tetel}{Kruskal Algoritmus}
Bemenet: G gráf és az élekhez tartozó w súlyfüggvény. Az éleket rendezzük sorba úgy, hogy a legalacsonyabb költségűek legyenek először a sorban. A sorban kezdjünk előre haladni. Ha az él bevétele esetén a kapott gráf körmentes marad, akkor vegyük be. Ezt addig ismételjük, amíg van izolált pont, vagy amíg az élsorozat végére nem érünk. A kapott gráf a G gráf minimális költségű feszítőfája.
\\
\\
Ezt az eljárást \textbf{mohó algoritmusnak} nevezzük, mivel a végrehajtás során minden lépésben az éppen akkor a legjobbnak tűnő lehetőséget választjuk ki.
\end{tetel}

\begin{definicio}{MINIMÁLIS SÚLYÚ FESZÍTŐFA}
Legyen G gráf és annak éleihez rendelt $w: E \rightarrow \mathbb{R}$ súlyfüggvény. A gráfnak azon feszítőfáját, melyre ez a súlyfüggvény minimális, a gráf \textbf{minimális súlyú feszítőfájának} nevezzük.
\end{definicio}

\begin{tetel}{}
A Kruskal-algoritmus minimális súlyú feszítőfát talál.
\end{tetel}

\begin{bizonyitas}{}
Két állítást kell bizonyítanunk - azt, hogy feszítőfát ad az algoritmus, és azt, hogy az minimális. Kezdjük az elsővel. Legyen G egy összefüggő, súlyozott gráf (tehát van súlyfüggvény hozzárendelve) és F legyen egy részgráfja, amit az algoritmus produkál. F-ben nem lehet kör, mivel az algoritmus egy fát épít. F nem lehet nem összefüggő sem, mivel az első él (amit az algoritmus talál), ami összeköt két független komponenst F-ben még nem hozhat létre kört. Tehát F feszítőgráf.
\\
A következő állítást teljes indukcióval bizonyítjuk be. Legyen H egy élhalmaz, amit az algoritmus a futása során generál, a minimális súlyú feszítőfának ezt a H élhalmazt tartalmaznia kell, hiszen ebben vannak minimális súlyú élek.
Az első lépésnél az állítás igaz, hiszen H üres, és minden gráfnak részgráfja az üres gráf. A k-adik lépésnél vegyük az állítást igaznak és legyen T a minimális súlyú feszítőfa, ami tartalmazza H-t. Ha az algoritmus által kiválasztott következő él, e, szintúgy benne van a T-ben, akkor az állítás szintúgy igaz a $H + {e}$ élhalmazra. Különben $T + {e}$ élhalmazban létezik egy C kör, és ezen kívül még létezik egy olyan f él, ami befejezi a C kört, de nem része H-nak. (Ha nem létezne f, akkor e-t már nem vehettük volna be, mivel kört produkált volna $H + f$-ben). Ekkor $T - {f} + {e}$ szintúgy egy fa, és azonos az összsúlya T-jével, hiszen T-nek minimális az összsúlya, és f-nek a súlya nem lehet kisebb, mint e-nek, hiszen akkor e helyett az f élet választotta volna az algoritmus. Tehát $T - {f} + {e}$ egy minimális súlyú feszítőfa. Ezek alapján az indukciós feltevést bebizonyítottuk, és az állítás igaz, amikor H egy feszítőfává válik, ami csak akkor igaz, ha H egy minimális súlyú feszítőfa.
\end{bizonyitas}

\begin{tetel}{Alkalmazás}
Pontok - városok, élek - utak, súly - hossz; Villamos hálózatok, Kirchhoff-törvények, áramköri elemekhez súlyokat párosítunk
\end{tetel}
