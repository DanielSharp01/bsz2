\section{3. tétel: Síkbarajzolhatóság}

\begin{definicio}{SÍKBARAJZOLHATÓSÁG}
G \textbf{síkbarajzolható} gráf, ha lerajzolható úgy (a síkba), hogy az élei a csúcsokon kívül sehol máshol ne keresztezzék egymást.
\end{definicio}

\begin{definicio}{TARTOMÁNYOK}
G síkbarajzolt gráf \textbf{tartományain} azon síkrészeket értjük, melyeket közrefognak az élek. Csak síkbarajzolt gráfok esetén beszélhetünk ezekről!
\end{definicio}

\begin{tetel}{GÖMBRE RAJZOLHATÓSÁG}
G gráf pontosan akkor síkbrajzolható, ha gömbre rajzolható.
\end{tetel}

\begin{bizonyitas}{}
Egy síkban lévő gráf leképezhető gömbfelületre oly módon, hogy ezt a gömbfelületet valamelyik pontjával a síkra helyezzük, ezt a pontot tekintjük a déli pólusként, és az északi pólusból egyeneseket húzunk a gráf pontjaiba. Ezeknek a vonalaknak van metszéspontja a gömbön, ezek szolgáltatják a kívánt vetítést. Ez az ú.n. sztereografikus projekció. Ezt visszafele is meg lehet ismételni.
\end{bizonyitas}

\begin{tetel}{EULER-FORMULA}
Ha egy összefüggő síkbeli gráfnak n csúcsa, e éle és t tartománya van (beleértve a külső tartományt is), akkor eleget tesz az Euler-formulának:
$$n + t = e + 2$$
\end{tetel}

\begin{bizonyitas}{}
Tekintsük a gráf egy C körét (ha van) és ennek egy $a$ élét. A C kör a síkot két részre osztja. Ezeket egyéb élek további tartományokra oszthatják, de mindkét részben van egy olyan tartomány, melynek $a$ a határa. Ha a-t elhagyjuk, a két tartomány egyesül, azaz a tartományok száma eggyel csökken. A csúcsok száma nem változik, tehát $a$ elhagyásával az $n - e + t$ érték nem változik. Ezt az eljárást addig folytassuk, amíg a gráfban nem marad kör. Ekkor viszont már csak egy feszítőfa maradt. Elég az állítást erre belátni, ami triviális, hiszen $t = 1$ és $e = n - 1$.
\end{bizonyitas}

\begin{tetel}{BECSLÉS AZ ÉLEK SZÁMÁRA}
Ha G egyszerű, síkbarajzolható gráf és pontjainak a száma legalább 3, akkor az előbbi jelölésekkel:
$$e \leq 3n - 6$$
\end{tetel}

\begin{bizonyitas}{}
Vegyük G tetszőleges síkbarajzolását és jelöljük az egyes tartományokat határoló élek számát $c_1, c_2...c_t$-vel. Mivel a gráf egyszerű, ezért minden tartományát legalább 3 él határolja, tehát $c_i \geq 3$. Nyilvánvaló, hogy egy élhez legfeljebb 2 tartomány tartozik, tehát ha összegezzük a tartományokat határoló élek számát minden tartományra, akkor legfeljebb $2e$-t kaphatunk. Tehát:
$$3t \leq c_1 + c_2 + ... + c_t = \sum_{i=1}^{t} c_i \leq 2e$$
Az Euler-formulát felhasználva:
$$3(e - n + 2) \leq 2e$$
Ebből átrendezéssel megkapjuk az eredményt.
\end{bizonyitas}

\begin{tetel}{BECSLÉS AZ ÉLEK SZÁMÁRA}
Ha G egyszerű, síkbarajzolható gráf és minden köre legalább 4 hosszú, valamint legalább 4 pontja van, akkor:
$$e \leq 2n - 4$$
\end{tetel}

\begin{bizonyitas}{}
Minden tartományt legalább 4 él határol. Az előző biz. gondolatmenete alapján $4t \leq 2e$ és ez alapján megkapjuk a képletet.
\end{bizonyitas}

\begin{tetel}{BECSLÉS MINIMÁLIS FOKSZÁMRA}
Ha G egyszerű, síkbarajzolható gráf, akkor $$\delta = min\, d(v) \leq 5$$ azaz a minimális fokszám legfeljebb 5.
\end{tetel}

\begin{bizonyitas}{}
Feltehetjük, hogy a gráf pontjainak a száma legalább 3. T.f.h. $\delta \geq 6$. Mivel a fokszámok összege egyenlő az élszámok kétszeresével, $6n \leq 2e$. Az élszám-becslés alapján azonban $2e \leq 6n - 12$, ezzel ellentmondásra jutottunk, mivel $6n \not\leq 6n - 12$.
\end{bizonyitas}

\begin{tetel}{KURATOWSKI-GRÁFOK}
A Kuratowski-gráfok, tehát a $K_5$ és a $K_{3,3}$ nem rajzolhatóak síkba.
\end{tetel}

\begin{bizonyitas}{}
Ha $K_5$ síkbarajzolható volna, akkor teljesülne rá az élbecslés tétel. Azonban $K_5$ pontjainak száma 5, éleinek száma 10 és $10 \not\leq 3 \cdot 5 -6 = 9$, tehát $K_5$ nem síkbarajzolható.
A $K_{3,3}$ minden körének hossza legalább 4. Ha volna 3 hosszú kör, legalább 2 ``kút'' vagy ``ház'' között kellene annak mennie, ami nem lehetséges. Így tehát a 2. élbecslés alkalmazható. $K_{3,3}$ pontjainak a száma 6, éleinek száma 9 és $9 \not\leq 2 \cdot 6 - 4 = 8$, tehát $K_{3,3}$ nem síkbarajzolható.
\end{bizonyitas}

Érdemes megjegyezni, hogy ezeknek a topologikus izomorf megfelelőit (tehát ha egy él helyett 2 hosszú út van) se lehet síkba lerajzolni. Ezeket úgy tudjuk konstruálni, hogy egy él helyett vagy egy új, 2-fokú csúccsal helyettesítünk, vagy egy 2-fokú csúcsot egy éllel.

\begin{definicio}{TOPOLOGIKUS IZOMORFIA}
  A G és H gráfok \textbf{topologikusan izomorfak}, ha a csúcsok élekről való elhagyásával vagy azokra való felvételével, ezek ismételt alkalmazásával izomorf gráfokba transzformálhatóak.
\end{definicio}

\begin{tetel}{KURATOWSKI-TÉTEL}
Egy gráf akkor és csak akkor síkbarajzolható, ha nem tartalmaz olyan részgráfot, amely topologikusan izomorf lenne $K_{3,3}$-al vagy $K_5$-el. \textbf{A bizonyítás a szükségességre fentebb található.}
\end{tetel}

\begin{definicio}{DUÁLIS}
Egy G gráf \textbf{duálisát} úgy kapjuk meg, hogy G tartományaihoz rendelünk pontokat (G* pontjai) és G*-ban akkor kötünk össze két pontot, ha a megfelelő G-beli tartományoknak van közös határvonala.
\end{definicio}

A duális gráfban tehát $n* = t$ és $t* = n$, valamint $e* = e$.
A párhuzamos élekből ``soros élek'', hurokélekből pedig elvágó élek lesznek.