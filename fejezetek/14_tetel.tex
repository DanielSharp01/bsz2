\section{14. tétel: DFS algoritmus}

\url{http://cs.bme.hu/bsz2/dfs.pdf}

\begin{tetel}{A DFS algoritmus a következő adatokat tartja számon}
\begin{itemize}
\item d(v): a v csúcs mélységi száma
\item f(v): a v csúcs befejezési száma
\item m(v): a v-t megelőző csúcs - tehát amiből a v-t a bejárás elérte
\item a: a jelenleg aktív csúcs
\item g: az aktuális gyökérpont
\item D: aktuális mélységi szám
\item F: az eddigi legnagyobb befejezési szám
\end{itemize}
\end{tetel}

\begin{tetel}{DFS Algoritmus}
Bemenet: Egy n csúcsú G irányított gráf és egy $s \in V$ csúcs.
\begin{itemize}
\item{\textbf{0.}} $d(s) = 1$, minden $v \neq s$-re $d(v) = *$, minden v-re $f(v) = *$, minden v-re $m(v) = *$, $a = s$, $g = s$, $D = 1$, $F = 0$.
\item{\textbf{1.}}
\\
HA létezik olyan $e = \overrightarrow{av}$ él, melyre $d(v) = *$, AKKOR:
	\begin{itemize}
	\item $D = D + 1$
	\item $d(v) = D$
	\item $m(v) = a$
	\item $a = v$
	\item \textbf{1.} lépéshez vissza
	\end{itemize}
\item{\textbf{2.}}
	\begin{itemize}
	\item $F = F + 1$
	\item $f(a) = F$
	\item HA $a \neq g$, AKKOR $a = m(a)$, $D = D - 1$ és \textbf{1.} lépéshez vissza
	\item HA $D = n$, AKKOR \textbf{STOP}.
	\item Válasszunk olyan v csúcsot, melyre $d(v) = *$.
	\item $g = v$, $a = v$, \textbf{1.} lépéshez vissza.
	\end{itemize}
\end{itemize}
\end{tetel}

\begin{tetel}{DFS Algoritmus, for coders}
Bemenet: Egy n csúcsú G irányított gráf és egy $s \in V$ csúcs.
\begin{itemize}
\item{\textbf{0. Inicializálás}}
  \begin{itemize}
  \item $[MelysegiSzam](s) = 1$
  \item minden $v \neq s$-re $[MelysegiSzam](v) = URES$
  \item minden v-re $[BefejezesiSzam](v) = URES$
  \item minden v-re $[MegelozoCsucs](v) = URES$
  \item $[AktivCsucs]=s$
  \item $[GyokerPont] = s$
  \item $[Melyseg] = 1$
  \item $[MaxBefejezesiSzam] = 0$
  \end{itemize}
\item{\textbf{1.}}
\\
HA létezik olyan $e = \overrightarrow{av}$ él, melyre $[MelysegiSzam](v) = URES$, AKKOR:
	\begin{itemize}
	\item $[Melyseg] = [Melyseg] + 1$
	\item $[MelysegiSzam](v) = [Melyseg]$
	\item $[MegelozoCsucs](v) = [AktivCsucs]$
	\item $[AktivCsucs] = v$
	\item \textbf{1.} lépéshez vissza
	\end{itemize}
\item{\textbf{2.}}
	\begin{itemize}
	\item $[MaxBefejezesiSzam] = [MaxBefejezesiSzam] + 1$
	\item $[BefejezesiSzam]([AktivCsucs]) = [MaxBefejezesiSzam]$
	\item HA $[AktivCsucs] \neq [GyokerPont]$, AKKOR $[AktivCsucs] = [MegelozoCsucs]([AktivCsucs])$; $[Melyseg] = [Melyseg] - 1$ és \textbf{1.} lépéshez vissza
	\item HA $[Melyseg] = n$, AKKOR \textbf{STOP}.
	\item Válasszunk olyan v csúcsot, melyre $[MelysegiSzam](v) = URES$.
	\item $[GyokerPont] = v$, $[AktivCsucs] = v$, \textbf{1.} lépéshez vissza.
	\end{itemize}
\end{itemize}
\end{tetel}

\begin{tetel}{DFS Algoritmus, for coders}
Bemenet: Egy n csúcsú G irányított gráf és egy $s \in V$ csúcs.
\begin{itemize}
\item{\textbf{0. INICIALIZÁLÁS}}
  \begin{itemize}
  \item Csinálunk egy üres táblázatot, ezekkel az adatokkal:\\
    
    \begin{tabular}{c l c}
      Jelölés & Jelentés & 1. oszlop\\
      \hline
      v & csúcs & s\\
      d & mélységi szám (kb. az iteráció száma) & 1\\
      f & befejezési szám (hanyadikként fejeztük be) & URES\\
      m & honnan derítettük fel ezt a csúcsot & -\\
      a & aktív csúcs mutató & $\uparrow$\\
    \end{tabular}

  \item Külön helyen (nem táblázatban) jegyezzük ezeket az adatokat:\\
    
    \begin{tabular}{c l c}
      Jelölés & Jelentés & Inicializálás\\
      \hline
      g & Gyökérpont (nem összefüggő gráfokra) & s\\
      D & Eddigi legnagyobb mélységi szám& 1\\
      F & Eddigi legnagyobb befejezési száma& 0\\
    \end{tabular}
  \end{itemize}
\item{\textbf{1. HALADÁS ELŐRE}}
  \\
  HA a jelenleg aktív csúcsból ($a$) vezet olyan él, aminek a mélységi száma ($m$) URES (vagyis a jelenleg aktív csúcsból tudunk haladni ``előre''), AKKOR:
  \begin{itemize}
  \item Megnöveljük eggyel a maximális mélységet
  \item A táblázatot bővítjük egy új oszloppal:\\

    \begin{tabular}{c l}
      Jelölés & Új oszlop\\
      \hline
      m & az előbb említett él elején lévő csúcs - jegyezzük, honnan jöttünk\\
      d & az új, megnövelt maximális mélység\\
    \end{tabular}

    (f-et még nem ismerjük, hagyjuk üresen)

  \item Az aktív csúcs nyilacskát tegyük eggyel jobbra.

	\item \textbf{1. HALADÁS ELŐRE} lépéshez vissza
	\end{itemize}
    KÜLÖNBEN nem tudunk tovább előre menni, ezért el kell indulnunk vissza; 2. lépés.
\item{\textbf{2. VISSZA}}
  \begin{itemize}
  \item ``Befejezzük'' a jelenlegi aktív csúcsot: inkrementáljuk a befejezési számot ($F$) és beírjuk ehhez a csúcshoz ($f$).
  \item Ha a jelenlegi aktív csúcs nem gyökérpont, tehát lehet belőle visszamenni, akkor menjünk is vissza: írjuk be az aktív csúcshoz a megelőző csúcsot, illetve csökkentsük a mélységet, és próbálkozzunk újra a haladással az 1. lépés szerint.
  \item Folytassuk a 3. lépésnél
  \end{itemize}
\item \textbf{3. NEM ÖSSZEFÜGGŐ A GRÁF}
  \begin{itemize}
  \item Azaz, ha minden csúcsot bejártunk már, akkor leállhatunk.
  \item KÜLÖNBEN Válasszunk olyan v csúcsot, melyre még ismeretlen a mélységi szám, tehát olyat, amit még nem jártunk be. Tekintsük ezt gyökérpontnak és tegyük oda az aktív csúcs nyilacskát. Folytassuk az 1. lépésnél.
  \end{itemize}
\end{itemize}

Ha a gráf egy DAG, akkor a \textbf{topologikus sorrend} a befejezési sorrend fordítottja.
\end{tetel}

\begin{definicio}{DFS-ERDŐ}
s csúcsból indítva G irányított gráfban lefuttattuk a DFS algoritmust. A futáshoz tartozó DFS erdő F. Legyen $e=\overrightarrow{uv}$ a G-nek tetszőleges éle. Ekkor
A BFS-erdő építéséhez hasonlóan megkaphatjuk a DFS-erdőt úgy, hogy az egyes mélységi szinteknek megfelelő fában helyezzük el a gráf pontjait, és berajzoljuk a gráfban meglévő éleket. Ebben az esetben utóbbiakat az alábbiak szerint osztályozhatjuk:
\begin{itemize}
\item faél: a DFS-fa része
\item előreél: ősből a leszármazottba mutat, de nem faél.\\
  Ez úgy állhat elő, hogy az algoritmus nem feltétlenül haladt végig a kérdéses élen, de az az irány is egy lehetősége volt egy bizonyos pontján az algoritmusnak.
\item visszaél: leszármazottból az ősbe mutat.
\item keresztél: olyan csúcsok között haladnak, amelyek nem leszármazottjai egymásnak.
\end{itemize}
\end{definicio}

\begin{tetel}{DFS ÉS ACIKLIKUSSÁG}
Ha G irányított gráf aciklikus, a DFS futtatásakor nem keletkezik visszél. Ha nincs visszél, akkor f(v) szerinti csökkenő sorrend a topologikus sorrend.
\end{tetel}

\begin{bizonyitas}{}
Ha keletkezik visszél és $e = \overrightarrow{uv}$ ilyen, akkor G irányított gráf nyilván tartalmaz irányított kört: az F DFS-erdő tartalmaz v-ből u-ba irányított utat (mert e visszél), amit e-vel kiegészítve irányított körré zárhatunk. Tegyük fel ezért, hogy nem keletkezik visszaél. Ha megmutatjuk a tétel 2. állítását, hogy a csúcsoknak a befejezési számozás szerinti fordított sorrendje topologikus rendezés, akkor ebből nyilván következik G aciklikussága is. Azt kell megmutatnunk, hogy G minden $e = \overrightarrow{uv}$ élére $f(v) < f(u)$. Mivel feltettük, hogy visszaél nem keletkezett, ezért e lehet faél, keresztél vagy előreél. Ha e faél vagy előreél, akkor a DFS eljárás során u első aktívvá válásakor az u befejezéséig tartó szakasza során érte el v-t, így ennek során kellett befejeznie azt, tehát a v-t előbb fejezte be u-nál, tehát $f(v) < f(u)$. Ha pedig e keresztél, akkor e vizsgálatának a pillanatában f(v) már kapott értéket, u viszont épp aktív csúcs volt, így ekkor még $f(u) = *$ volt. Mivel az eljárás egyre nagyobb befejezési számokat ad, ezért $f(v) < f(u)$ erre is teljesülni fog.
\end{bizonyitas}
