\section{10. tétel: Menger tételei}

\begin{definicio}{DISZJUNKT UTAK}
Vegyük (G,s,t,c) folyamot, ezen belül veszünk utakat. \textbf{Páronként éldiszjunkt vagy élidegen útnak} nevezünk utakat, ha páronként nincsen közös élük. \textbf{Belsőleg pontdiszjunkt utaknak} nevezünk utakat, ha páronként nincs közös pontjuk.
\end{definicio}

\begin{tetel}{MENGER Tétel}
Ha G egy irányított gráf, $s,t\in V(G)$, akkor az s-ből t-be vezető páronként élidegen irányított utak maximális száma megegyezik az összes irányított s-t utat lefogó élek minimális számával.
\end{tetel}

\begin{bizonyitas}{}
Ha létezik G-ben k darab ilyen irányított s-t út, akkor az s-t utakat lefogó élek száma nyilvánvalóan legalább k. Nézzük az egyenlőtlenség másik oldaláról is. T.f.h. az s-t utakat lefogó élek minimális száma k. Legyen minden él kapacitása 1. Az így kapott hálózatban a maximális folyam értéke legalább k. Ekkor a Ford-Fulkerson tétel miatt a minimális vágás értéke is legalább k. Azt már beláttuk, hogy van olyan maximális folyam, ahol minden élen a folyamérték 0 vagy 1. Lássuk be, hogy G-ben van k élidegen irányított s-t út. Egy ilyen út legalább van, különben nem lehetne k a folyam értéke. Az ebben az útban szereplő élek kapacitását csökkentsük nullára, így a folyam értéke legalább k-1 lesz. Folytassuk a gondolatmenetet és kapunk k élidegen irányított s-t utat.
\end{bizonyitas}

\begin{tetel}{MENGER Tétel}
Ha G egy irányított gráf, $s,t\in V(G)$ két nem szomszédos pont, akkor az s-ből t-be vezető, végpontoktól eltekintve pontidegen irányított utak maximális száma megegyezik az összes irányított s-t utat s és t felhasználása nélkül lefogó pontok minimális számával.
\end{tetel}

\begin{bizonyitas}{}
Készítsünk egy új G' gráfot. Minden pontot húzzunk szét két ponttá. Ha a G gráfban egy minimális ponthalmaz lefogja az irányított s-t utakat, akkor a lefogó pontoknak megfelelő (v', v'') pontok is lefogó ponthalmazt alkotnak s-t-re G'-ben. Kevesebb él nem elég a lefogáshoz, mert ha a lefogó élek közt lennének (a'', b') típusú élek, akkor ezeket helyettesítjük (b', b'')-vel, ha $b' \neq t$, illetve (a', a'')-val, ha $b' = t$. Így pedig G-ben egy kisebb lefogó ponthalmazt nyernénk. Vagyis a G-beli lefogó pontok és a G'-beli lefogó élek minimális száma egyenlő. Az is könnyen látható, hogy G-beli pontdiszjunkt utaknak G'-ben éldiszjunkt utak felelnek meg és fordítva, G'-beli élidegen utaknak G-ben pontidegen utaknak felelnek egy. Innen az előző tétel segítségével bizonyítjuk az állítást.
\end{bizonyitas}

\begin{tetel}{MENGER Tétel}
Ha G egy irányítatlan gráf, $s,t\in V(G)$, akkor az s-ből t-be vezető élidegen irányítatlan utak maximális száma megegyezik az összes irányítatlan s-t utat lefogó élek minimális számával.
\end{tetel}

\begin{bizonyitas}{}
Vezessük vissza irányított gráfos problémára. A bizonyítás a számításelmélet jegyzet 70. oldalán található, hosszú.
\end{bizonyitas}

\begin{tetel}{MENGER Tétel}
Ha G egy irányítatlan gráf, $s,t\in V(G)$ két nem szomszédos pont, akkor az s-ből t-be vezető, végpontoktól eltekintve pontidegen irányítatlan utak maximális száma megegyezik az összes irányítatlan s-t utat s és t felhasználása nélkül lefogó pontok minimális számával.
\end{tetel}

\begin{bizonyitas}{}
Előző tételhez hasonlóan.
\end{bizonyitas}

\begin{definicio}{TÖBBSZÖRÖS ÖSSZEFÜGGŐSÉG}
Egy G gráfot \textbf{k-szorosan összefüggőnek} nevezünk, ha legalább k+1 pontja van, és akárhogy hagyunk el belőle k-nál kevesebb pontot, a maradék gráf összefüggő marad. A gráf \textbf{k-szorosan élösszefüggő}, ha akárhogy hagyunk el belőle k-nál kevesebb élt, összefüggő gráfot kapunk. A k-szoros összefüggőség ``erősebb'' a k-szoros élösszefüggőségnél.
\end{definicio}

\begin{tetel}{EKVIVALENCIA PONT- ÉS ÉLÖSSZEFÜGGŐSÉGRE}
A G gráf akkor és csak akkor k-szorosan összefüggő, ha legalább $k + 1$ pontja van, és bármely két pontja között létezik k pontidegen út.
Hasonlóan, G akkor és csak akkor k-szorosan élösszefüggő, ha bármely két pontja között létezik k élidegen út.
\end{tetel}

\begin{bizonyitas}{}
Először a második részt bizonyítjuk. Ha G k-szorosan élösszefüggő, akkor az u-v utakat lefogó élek minimális száma nyilván legalább k. Így Menger idevágó tétele szerint élidegen u-v utak maximális száma legalább k. Ennek a résznek a megfordítása is következik a Menger tételből. Ha G k-szorosan összefüggő, akkor bármely két, $u,v\in V(G)$ pontot választva legalább k darab, u-tól és v-től különböző pontra van szükség ahhoz, hogy lefogjuk az összes u és v közötti utat (u-v éltől eltekintve). Így az utolsó Menger tétel alapján létezik u és v között k pontidegen út. Ha G bármely két pontja között létezik k pontidegen út, akkor nyilván nem lehet ezeket k-nál kevesebb ponttal lefogni, tehát a k-szoros összefüggőség következik ebből.
\end{bizonyitas}

\begin{tetel}{MENGER Tétel}
A legalább 3 pontú G gráf akkor és csak akkor 2-szeresen összefüggő, ha tetszőleges két pontján át vezet kör. Igaz az is, hogy akkor és csak akkor 2-szeresen összefüggő, ha bármely két élén át vezet kör.
\end{tetel}

\begin{bizonyitas}{}
Az első állítás triviális, hiszen két pontidegen u-v út együtt egy kört ad, amely átmegy u-n és v-n. A második állítás pedig az elsőből következik. Lássuk be, hogy ha G 2-szeresen összefüggő, akkor e, f éleken keresztül van kör. Vegyünk fel két pontot úgy, hogy ezekkel osszuk két részre az e illetve az f élt. Az így kapott gráf is 2-szeresen összefüggő. Az első állítás szerint ezen a két ponton át is megy kör, és ez a kör az eredeti gráfban átmegy e-n és f-en. A megfordítás ismét nyilvánvaló.
\end{bizonyitas}
