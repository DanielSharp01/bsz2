\section{1. tétel: Kombinatorika}

\begin{definicio}{FAKTORIÁLIS}
Az $n(n-1)(n-2)...\cdot2\cdot1$ szorzatot n \textbf{faktoriálisának} nevezzük. Definíció szerint $0! = 1$.\\
Jel: $n!$
\end{definicio}

\begin{definicio}{PERMUTÁCIÓ}
Az n elem összes lehetséges sorrendjének a száma $n!$ és ezt hívjuk \textbf{permutációnak}.
\end{definicio}

\begin{definicio}{ISMÉTLÉSES PERMUTÁCIÓ}
$k_1$ darab első típusú elem, ..., $k_n$ darab n-edik típusú elem lehetséges sorba rendezésének a száma a $k_1 + k_2 + ... + k_n$ \textbf{ismétléses permutációi}. Számuk:
$$\frac{(k_1+k_2+...+k_n)!}{k_1!\cdot k_2!\cdot...\cdot k_n!}$$
\end{definicio}

\begin{definicio}{VARIÁCIÓ}
n-ből k elem összes lehetséges sorrendben való kiválasztása az n elem k-ad osztályú (ismétlés nélküli) \textbf{variációja}, ezek száma:
$$n(n-1)(n-2)...(n-k+1) = \frac{n!}{(n-k)!}$$
Ha $k=n$, akkor permutációról beszélünk.
\end{definicio}

\begin{definicio}{ISMÉTLÉSES VARIÁCIÓ}
n elemből k tagú sorozatok kiválasztása, ahol egy-egy elem többször is szerepelhet, az n-elem k-ad osztályú \textbf{ismétléses variációja}. Ezeknek a száma:
$$n^k$$
\end{definicio}

\begin{definicio}{KOMBINÁCIÓ}
Egy n elemű halmaz k elemű részhalmazainak a száma: n elem k-ad osztályú \textbf{kombinációja}. Száma:
$$\begin{pmatrix}
n\\k
\end{pmatrix} = \frac{n(n-1)(n-2)...(n-k+1)}{k!}$$
Az $\begin{pmatrix}
n\\k
\end{pmatrix}$ a binomiális együttható.
\end{definicio}

\begin{definicio}{ISMÉTLÉSES KOMBINÁCIÓ}
n elemből k kiválasztása, ha a sorrend nem számít, de az elemek többször is szerepelhetnek: n elem k-ad osztályú \textbf{ismétléses kombinációi}. Számuk: $$\begin{pmatrix}
n+k-1\\k
\end{pmatrix}$$
\end{definicio}

\begin{tetel}{BINOMIÁLIS TÉTEL}
  Tetszőleges valós x, y-ra és nemnegatív egész n-re:
  \[ (x+y)^n = \sum_{k=0}^{n} \binom{n}{k} x^{n-k}y^{k} {\color{gray} = \binom{n}{0}x^n + \binom{n}{1}x^{n-1}y^1 + \binom{n}{2}x^{n-2}y^2 + \dots + \binom{n}{n}y^n } \]
\end{tetel}

\begin{definicio}{PASCAL HÁROMSZÖG}
A háromszög csúcsán álljon az  $\begin{pmatrix}
0\\0
\end{pmatrix}$ együttható. Alatta álljon az $\begin{pmatrix}
1\\0
\end{pmatrix}$ illetve a $\begin{pmatrix}
1\\1
\end{pmatrix}$ együttható. A piramis $(k-1)$-edik sorában álljanak a $\begin{pmatrix}
k\\0
\end{pmatrix}$, $\begin{pmatrix}
k\\1
\end{pmatrix}$ ... $\begin{pmatrix}
k\\k-1
\end{pmatrix}$, $\begin{pmatrix}
k\\k
\end{pmatrix}$ együtthatók. A legutóbbi állítás alapján a Pascal-háromszög minden sorának a sorösszege:
$2^{k-1}$. Ez abból is belátható, hogy minden sor összege kétszerese az előzőnek, ugyanis az együtthatókat úgy is meg lehet kapni, hogy az új elem a felette álló két együttható összegéből áll össze.
\end{definicio}

\begin{tetel}{BINOMIÁLIS ÖSSZEG}
Minden n nemnegatív számra
$$\begin{pmatrix}
n\\0
\end{pmatrix} + \begin{pmatrix}
n\\1
\end{pmatrix} + \begin{pmatrix}
n\\2
\end{pmatrix} + ... + \begin{pmatrix}
n\\n-1
\end{pmatrix} + \begin{pmatrix}
n\\n
\end{pmatrix} = 2^n = \sum_{k = 0}^{n}\begin{pmatrix}
n\\k
\end{pmatrix}$$
\end{tetel}

\begin{tetel}{ÖSSZEFÜGGÉSEK A BINOMIÁLIS EGYÜTTHATÓK KÖZÖTT}
$$
\begin{pmatrix}
n\\k
\end{pmatrix} = \begin{pmatrix}
n\\n-k
\end{pmatrix}
$$
$$
\begin{pmatrix}
n\\k
\end{pmatrix} = \begin{pmatrix}
n-1\\k-1
\end{pmatrix} + \begin{pmatrix}
n-1\\k
\end{pmatrix}
$$
\end{tetel}
