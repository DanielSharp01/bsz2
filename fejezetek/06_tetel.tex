\section{6. tétel: $\nu, \rho, \alpha, \tau$}

\begin{definicio}{PÁROSÍTÁS}
  \begin{itemize}
  \item \textbf{Független élhalmaz} vagy \textbf{(Részleges/Teljes) Párosítás}: olyan élhalmaz, amiben semelyik két élnek nincsen közös pontja.
  \item A párosítás éleinek végpontjait \textbf{lefedi}.
    \begin{itemize}
    \item Teljes párosítás: lefedi a gráf összes csúcsát.
    \item Részleges párosítás: nem :)
    \end{itemize}
  \end{itemize}
  \imgkozep{parositas}
\end{definicio}

\begin{definicio}{FÜGGETLEN/LEFOGÓ ÉLEK/PONTOK HALMAZA (``görög betűk'')}
  \begin{itemize}
  \item Jelöljük $\nu(G)$-vel a G gráfban található \textbf{független élek} maximális számát.
  \item $Y \subseteq E(G)$ \textbf{lefogó élhalmaz}, ha minden pontot lefog.\\
    A lefogó élek minimális számát $\rho(G)$ jelöli.
  \item $X \subseteq V(G)$ \textbf{független ponthalmaz}, ha nincs benne két szomszédos pont.\\
    A független pontok maximális száma $\alpha(G)$
  \item $X \subseteq V(G)$ egy \textbf{lefogó ponthalmaz}, ha G minden élének legalább egyik végpontját tartalmazza.\\
    A lefogó pontok minimális számát $\tau(G)$-vel jelöljük.
  \end{itemize}
  \imgkozep{gorogbetuk}
\begin{center}
  \begin{tabular}{l c c}
    & Legnagyobb független & Legkisebb lefogó\\
    Ponthalmaz & $\alpha(G)$ & $\tau(G)$\\
    Élhalmaz & $\nu(G)$ & $\rho(G)$
  \end{tabular}
\end{center}
\end{definicio}

\begin{tetel}{``Görög betűk'' viszonya egymáshoz}
  Minden G gráfra
  
  \begin{minipage}{0.2\textwidth}
    $$\nu(G) \leq \tau(G)$$
    $$\alpha(G) \leq \rho(G)$$
  \end{minipage}
  \begin{minipage}{0.6\textwidth}
    $$\text{független élek maximális száma} \leq \text{lefogó pontok minimális száma}$$
    $$\text{független pontok maximális száma} \leq \text{lefogó élek minimális száma}$$
  \end{minipage}
\end{tetel}

\begin{bizonyitas}{}
Legyen M maximális méretű független élhalmaz. Mivel pusztán M éleinek lefogásához legalább $|M|$ pontra van szükség, ezért $\tau(G) \geq |M| = \nu(G)$.
Hasonlóan bizonyítjuk a második állítást is.
\end{bizonyitas}

\begin{tetel}{GALLAI TÉTEL I.}
Minden olyan G gráfra, mely hurokmentes:
$$\tau(G) + \alpha(G) = v(G) = n$$
\end{tetel}

\begin{bizonyitas}{}
Egy X halmaz pontjai akkor és csak akkor függetlenek, ha a $V(G)\backslash X$ halmaz lefogó ponthalmaz. Hiszen ha X nem független, akkor van két összekötött pont, és így $V(G)\backslash X$ nem fogja le ezt az élt. Fordítva, ha $V(G)\backslash X$ nem fog le egy huroktól különböző élt, akkor X-ben ennek az élnek mindkét végpontja szerepel. Tehát $\tau(G) \leq |V(G)\backslash X|$ teljesül minden X független ponthalmazra. Ebből pedig következik, hogy a $\tau(G) + \alpha(G) \leq v(G)$. Hasonlóan $\alpha(G) \geq |V(G) \backslash Y|$ minden Y lefogó ponthalmazra, amiből $\tau(G) + \alpha(G) \geq v(G)$ következik.
\end{bizonyitas}

\begin{tetel}{GALLAI TÉTEL II.}
Minden olyan G gráfra, melyben nincs izolált pont:
$$\nu(G) + \rho(G) = v(G) = n$$
\end{tetel}

\begin{bizonyitas}{}
Egy $\nu(G)$ elemű X független élhalmaz lefog $2\nu(G)$ különböző pontot. A többi pont (mivel nincs izolált) nyilván lefogható $v(G) - 2\nu(G)$ éllel, így $v(G) - \nu(G) \geq \rho(G)$. Másrészt, ha Y egy minimális lefogó élhalmaz, akkor Y néhány (k darab) diszjunkt csillag egyesítése. Ha ugyanis Y tartalmazna kört, akkor annak bármely élét, ha pedig 3 hosszú utat, akkor a közepét el lehetne hagyni Y-ból, mert a többi él még mindig lefogná az összes pontot. Így $\rho(G) = v(G) - k$. Minden csillagból kiválasztunk egy élt, az így kapott élhalmaz nyilván független. Tehát $\nu(G) \geq k = v(G) - \rho(G)$.
\end{bizonyitas}

\begin{tetel}{TUTTE-TÉTEL}
Egy G gráfban akkor és csak akkor van teljes párosítás, ha minden $X \subseteq V(G)$-re $c_p(G - X)\leq |X|$, azaz akárhogy hagyunk el a gráfból k pontot, a maradékban a páratlan komponensek száma nem lehet több, mint k.
\end{tetel}

\begin{bizonyitas}{}
(csak szükséges): Ha G-ben van teljes párosítás, akkor nyilvánvalóan teljesül a feltétel. Hiszen ha elhagyunk a gráfból X-et, akkor a páratlan komponensek mindegyikéből az eredeti gráfban indul ki legalább egy párosításbeli él, és ezek az élek csak egy-egy (különböző) X-beli pontba mehetnek. Tehát $c_p(G - X) \leq |X|$.
\end{bizonyitas}
