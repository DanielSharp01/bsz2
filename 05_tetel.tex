\documentclass[]{article}
\usepackage{lmodern}
\usepackage{amssymb}
\usepackage{amsmath}
\usepackage{polyglossia}
\usepackage{listings}
\usepackage{tcolorbox}
\usepackage{etoolbox}
\usepackage{setspace}
\usepackage{framed}
\usepackage[a4paper,margin=2cm,footskip=.5cm]{geometry}
\newcommand{\R}{\mathbb{R}}
\newcommand{\Rn}[1]{$\mathbb{R}^{#1}$}
\newcommand{\Und}[1]{\underline{#1}}
\definecolor{shadecolor}{gray}{0.9}
%opening 
\title{Bevezetés a Számításelméletbe 2.\\{\large 5. tétel}}
\author{Hegyi Zsolt}
\begin{document}
\maketitle
\begin{framed}
KURATOWSKI-GRÁFOK Tétel: A Kuratowski-gráfok, tehát a $K_5$ és a $K_{3,3}$ nem rajzolhatóak síkba.
\end{framed}
\begin{leftbar}
Ha $K_5$ síkbarajzolható volna, akkor teljesülne rá a élbecslés tétel. Azonban $K_5$ pontjainak száma 5, éleinek száma 10 és $10 \not\leq 3 \cdot 5 -6 = 9$, tehát $K_5$ nem síkbarajzolható.
A $K_{3,3}$ minden körének hossza legalább 4. Ha volna 3 hosszú kör, legalább 2 "kút" vagy "ház" között kellene annak mennie, ami nem lehetséges. Így tehát a 2. élbecslés alkalmazható. $K_{3,3}$ pontjainak a száma 6, éleinek száma 9 és $9 \not\leq 2 \cdot 6 - 4 = 8$, tehát $K_{3,3}$ nem síkbarajzolható.
\end{leftbar}
Érdemes megjegyezni, hogy ezeknek a topologikus izomorf megfelelőit (tehát ha egy él helyett 2 hosszú út van) se lehet síkba lerajzolni. Ezeket úgy tudjuk konstruálni, hogy egy él helyett vagy egy új, 2-fokú csúccsal helyettesítünk, vagy egy 2-fokú csúcsot egy éllel.
\begin{shaded}
TOPOLÓGIKUS IZOMORFIA Definíció: A G és H gráfok \textbf{topológikusan izomorfak}, ha a fentebb említett transzformációk ismételt alkalmazásával izomorf gráfokba transzformálhatóak.
\end{shaded}
\begin{framed}
KURATOWSKI-TÉTEL Tétel: Egy gráf akkor és csak akkor síkbarajzolható, ha nem tartalmaz olyan részgráfot, amely topológikusan izomorf lenne $K_{3,3}$-al vagy $K_5$-el. \textbf{A bizonyítás a szükségességre fentebb található.}
\end{framed}
\begin{framed}
FÁRY-WAGNER TÉTEL Tétel: Ha G egy egyszerű, síkbarajzolható gráf, akkor létezik olyan síkbeli ábrázolása is, melyben minden élet egy egyenes szakasszal rajzoltunk le.
\end{framed}
\begin{shaded}
DUÁLIS Definíció: Egy G gráf \textbf{duálisát} úgy kapjuk meg, hogy G tartományaihoz rendelünk pontokat (G* pontjai) és G*-ban akkor kössünk össze két pontot, ha a megfelelő G-beli tartományoknak van közös határvonala.
\end{shaded}
A duális gráfban tehát $n* = t$ és $t* = n$, valamint $e* = e$.
A párhuzamos élekből "soros élek", hurokélekből pedig elvágó élek lesznek.
\begin{shaded}
ELVÁGÓ ÉLHALMAZ Definíció: G összefüggő gráf, $x \in E(G)$. x elvágó élhalmaz, ha $(V(G),E(G)\backslash x) = G'$ és G' nem összefüggő. x vágás, ha x elv. élhalmaz, de semelyik részhalmaza sem az.
\end{shaded}
\begin{framed}
VÁGÁSOK ÉS KÖRÖK Tétel: G összefüggő és síkbarajzolt, ekkor ha C kör a G-ben, a C* vágás lesz G duálisában, G*-ban és fordítva, ha C vágás G-ben, a C* kör lesz a G*-ban.
\end{framed}
\begin{framed}
FA DUÁLISON BELÜL Tétel: G összefüggő, egyszerű gráf, F ezen belül feszítőfa. A G* duálison belül ez az F a saját komplementereként fog megjelenni.
\end{framed}
\end{document}
