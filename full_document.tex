\documentclass[]{article}
\usepackage{lmodern}
\usepackage{amssymb}
\usepackage{amsmath}
\usepackage{polyglossia}
\usepackage{listings}
\usepackage{tcolorbox}
\usepackage{etoolbox}
\usepackage{setspace}
\usepackage{framed}
\usepackage{hyperref}
\usepackage[a4paper,margin=2cm,footskip=.5cm]{geometry}
\newcommand{\R}{\mathbb{R}}
\newcommand{\Rn}[1]{$\mathbb{R}^{#1}$}
\newcommand{\Und}[1]{\underline{#1}}
\definecolor{shadecolor}{gray}{0.9}
\title{Bevezetés a számításelméletbe 2. tételek}
\author{Zsolt Hegyi}
\begin{document}
\maketitle
\section{tétel}
\input{01_tetel_stripped}
\newpage
\section{tétel}
\input{02_tetel_stripped}
\newpage
\section{tétel}
\input{03_tetel_stripped}
\newpage
\section{tétel}
\input{04_tetel_stripped}
\newpage
\section{tétel}
\input{05_tetel_stripped}
\newpage
\section{tétel}
\input{06_tetel_stripped}
\newpage
\section{tétel}
\input{07_tetel_stripped}
\newpage
\section{tétel}
\input{08_tetel_stripped}
\newpage
\section{tétel}
\input{09_tetel_stripped}
\newpage
\section{tétel}
\input{10_tetel_stripped}
\newpage
\section{tétel}
\input{11_tetel_stripped}
\newpage
\section{tétel}
\input{12_tetel_stripped}
\newpage
\section{tétel}
\input{13_tetel_stripped}
\newpage
\section{tétel}
\input{14_tetel_stripped}
\newpage
\section{tétel}
\input{15_tetel_stripped}
\newpage
\section{tétel}
\input{16_tetel_stripped}
\section{tétel}
\input{17_tetel_stripped}
\end{document}