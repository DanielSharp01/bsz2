\documentclass[]{article}
\usepackage{lmodern}
\usepackage{amssymb}
\usepackage{amsmath}
\usepackage{polyglossia}
\usepackage{listings}
\usepackage{tcolorbox}
\usepackage{etoolbox}
\usepackage{setspace}
\usepackage{framed}
\usepackage{hyperref}
\hypersetup{colorlinks=true}
\usepackage[a4paper,margin=2cm,footskip=.5cm]{geometry}
\newcommand{\R}{\mathbb{R}}
\newcommand{\Rn}[1]{$\mathbb{R}^{#1}$}
\newcommand{\Und}[1]{\underline{#1}}
\definecolor{shadecolor}{gray}{0.9}
%opening 
\title{Bevezetés a Számításelméletbe 2.\\{\large 3. tétel}}
\author{Hegyi Zsolt}
\begin{document}
\maketitle{}
\begin{framed}
ALGORITMUS: BFS $\rightarrow$ \url{http://cs.bme.hu/bsz2/bfs.pdf}
\end{framed}
\begin{framed}
ALGORITMUS: Kruskal $\rightarrow$ Az éleket rendezzük sorba úgy, hogy a legalacsonyabb költségűek legyenek először a sorban. A sorban kezdjünk előre haladni. Ha az él bevétele esetén a kapott gráf körmentes marad, akkor vegyük be. Ha az élsorozat végére érünk, akkor készen vagyunk. A kapott gráf a G gráf minimális költségű feszítőfája.
\\
\\
Ezt az eljárást \textbf{mohó algoritmusnak} nevezzük, mivel a végrehajtás során minden lépésben az éppen akkor a legjobbnak tűnő lehetőséget választjuk ki.
\end{framed}
\begin{framed}
Tétel: A Kruskal-algoritmus minimális súlyú fesz.fát talál.
\end{framed}
\begin{leftbar}
???
\end{leftbar}
\end{document}