\documentclass[]{article}
\usepackage{lmodern}
\usepackage{amssymb}
\usepackage{amsmath}
\usepackage{polyglossia}
\usepackage{listings}
\usepackage{tcolorbox}
\usepackage{etoolbox}
\usepackage{setspace}
\usepackage{framed}
\usepackage[a4paper,margin=2cm,footskip=.5cm]{geometry}
\newcommand{\R}{\mathbb{R}}
\newcommand{\Rn}[1]{$\mathbb{R}^{#1}$}
\newcommand{\Und}[1]{\underline{#1}}
\definecolor{shadecolor}{gray}{0.9}
%opening 
\title{Bevezetés a Számításelméletbe 2.\\{\large 2. tétel}}
\author{Hegyi Zsolt}
\begin{document}
\maketitle{}
\begin{shaded}
GRÁF Definíció: Egy \textbf{gráf} rendezett pár, $G = (V,E)$, ahol V egy nem üres halmaz, E pedig az ebből a halmazból képezhető párok egy halmaza, V elemeit \textbf{pontoknak} v. \textbf{csúcsoknak} nevezzük, E elemeit pedig \textbf{éleknek}. A csúcsok számát v(G)-vel jelöljük, az élekét pedig e(G)-vel. A gráf csúcshalmazát V(G)-vel, az élhalmazt pedig E(G)-vel jelöljük.
\end{shaded}
\begin{shaded}
HUROKÉL, TÖBBSZÖRÖS ÉL Definíció: Ha az $e \in E$ a {$v_1, v_2$} párnak felel meg, akkor \textbf{hurokélnek} nevezzük azon éleket, melynek két végpontja ugyanazon pont (tehát $v_1 = v_2$). Ha két nem különböző nem hurokélnek végpontjai azonosak, akkor a két élet \textbf{párhuzamos} v. \textbf{többszörös élnek} nevezzük.
\end{shaded}
\begin{shaded}
EGYSZERŰ GRÁF Definíció: Azokat a gráfokat, melyek nem tartalmaznak hurokéleket és többszörös éleket, \textbf{egyszerű gráfnak} nevezünk.
\end{shaded}
%\begin{shaded}
%SZOMSZÉDOS ÉLEK, PONTOK Definíció: Két él akkor \textbf{szomszédos}, ha legalább egy pontban megosztoznak. Hasonlóan, két pont akkor szomszédos, ha él összeköti őket. Ha egy pont egy él végpontja, akkor azt mondjuk, hogy a pont \textbf{illeszkedik} az élre.
%\end{shaded}
%\begin{shaded}
%FOKSZÁM Definíció: Egy pontra illeszkedő élek számát a pont \textbf{fokszámának} nevezzük. Amennyiben egy pont fokszáma 0, akkor azt \textbf{izolált pontnak} nevezzük. A v pont fokszámát d(v)-vel jelöljük. A maximális fokszámot $\Delta$-val (nagy delta), a minimálisat pedig $\delta$-val (kis delta) jelöljük. 
%\end{shaded}
\begin{shaded}
KOMPLEMENTER GRÁF Definíció: Egy G gráf \textbf{komplementerén} azt a $\bar{G}$ gráfot értjük, amelyet akkor kapunk, ha a G-t a $K_{v(G)}$ részgráfjának tekintjük és $\bar{G}$-ben azon pontpárok vannak összekötve, amelyek G-ben nincsenek.
\end{shaded}
\begin{shaded}
IZOMORFIA Definíció: G = (V,E) és G'= (V',E') gráfok \textbf{izomorfak}, ha van olyan egyértelmű megfeleltetés (bijekció), hogy G-ben pontosan akkor szomszédos két pont, a G'-ben a nekik megfelelő pontok szomszédosak, és a szomszédos pontpárok esetén ugyanannyi él fut közöttük.
\end{shaded}
\begin{shaded}
RÉSZGRÁF Definíció: A G' = (V',E') gráf a G = (V,E) \textbf{részgráfja}, ha a $V' \subseteq V$, $E' \subseteq E$ valamint egy pont és egy él pontosan akkor illeszkedik egymásra G'-ben, ha a G-ben is illeszkedők.
\end{shaded}
\begin{shaded}
FESZÍTŐ RÉSZGRÁF Definíció: G' = (V',E') gráf a G = (V,E)\textbf{ feszítő részgráfja}, ha G' részgráfja G-nek és V' = V.
\end{shaded}
\begin{shaded}
FESZÍTETT RÉSZGRÁF Definíció: Ha E' pontosan azokból az E-beli élekből áll, amelyeknek két végpontja V'-ben van, és az E' az összes ilyen élet tartalmazza, akkor G' a G gráf V' által feszített részgráfja.
\end{shaded}
\begin{shaded}
ÉLSOROZAT, ÚT, KÖR Definíció: Egy $(v_0, e_1, v_1 ... v_{k-1}, e_k, v_k)$ sorozatot \textbf{élsorozatnak} nevezzük, ha $e_i$ a $v_{i-1}$-et és $v_i$-t összekötő él. Ha $v_0 = v_k$, akkor az élsorozat zárt. Ha a csúcsok mind különbözőek, akkor egy \textbf{útról} beszélünk. Ha a csúcsok mind különbözőek és az élsorozat zárt, akkor pedig egy \textbf{körről}.
\end{shaded}
%\begin{framed}
%EKVIVALENCIA RELÁCIÓ Tétel: Definiáljuk úgy a $p \equiv q$ relációt, hogy $p \equiv q$ akkor és csak akkor, ha $p,q\in V(G)$ és vezet út p és q közt, vagy p = q, ez egy ekvivalencia reláció.  
%\end{framed}
%\begin{leftbar}
%Bizonyítás: Reflexivitás és szimmetria triviális. Tranzitivitás: Legyen két utunk, p-t q-val (($p, e_0, p_1, e_1...q$)) és q-t r-rel (($q, f_0, q_1, f_1...r$)) összekötő út. Legyen $p_i$ a legkisebb indexű p, ami előfordul q-k között is. Mondjuk, hogy $p_i = q_j$. Ekkor a ($p, e_0, p_1, e_1...p_i, f, q_{j+1},...r$) egy p-t és r-t összekötő út.
%\end{leftbar}
\begin{shaded}
ÖSSZEFÜGGŐSÉG Definíció: G gráf \textbf{összefüggő}, ha bármely 2 csúcs között létezik élsorozat vagy út.
\end{shaded}
\begin{shaded}
KOMPONENS Definíció: G gráf komponense olyan H feszített részgráfja G-nek, amire teljesül, hogy H összefüggő és bárhogy egy további csúcsot hozzáteszink, már nem összefüggő feszített részgráfot kapunk.
\end{shaded}
\begin{shaded}
FA Definíció: Az összefüggő körmentes gráfokat \textbf{fának} nevezzük. Amennyiben körmentes egy gráf, de nem összefüggő (tehát több fából áll), akkor azt \textbf{erdőnek} nevezzük.
\end{shaded}
\begin{framed}
FA FOKSZÁMA Tétel: Minden legalább 2 pontú fában van legalább 2 egy fokszámú pont.
\end{framed}
\begin{leftbar}
Tekintsük a fában található leghosszabb utat, legyen ez ($v_1, v_2,..., v_k$), ekkor beláthatjuk, hogy $v_1$ és $v_k$ végpontok egyfokúak lesznek. T.f.h $v_k$ nem egy fokszámú, tehát belőle vezet még él a fa valamely pontjába. Az út többi pontjába nem vezethet, mivel a fa definíció alapján körmentes, és új pontba sem vezethet, mivel akkor egy hosszabb utat kapnánk, amivel ellentmondást kapunk.
\end{leftbar}
\begin{framed}
FA ÉLEINEK SZÁMA Tétel: n pontú fa éleinek száma n-1.
\end{framed}
\begin{leftbar}
Bizonyítás: Teljes indukcióval. n = 2-re az állítás triviálisan teljesül. T.f.h. az állítás igaz minden $n < n_0$-ra. Az előző tétel szerint minden $n_0$ fában van egy fokszámú pont, ezt hagyjuk el. Ha elhagyjuk ezt a pontot és a hozzá tartozó élet, akkor mivel a maradék $n_0 - 1$ fára már igaz az állítás, látható, hogy az $n_0$ pontú eredeti fának $n_0 - 1$ éle van.
\end{leftbar}
\begin{shaded}
Az F gráf a G gráf feszítőfája, ha F fa, és részgráfja G-nek.
\end{shaded}
\begin{framed}
FESZÍTŐFA LÉTEZÉSE Tétel: Minden összefüggő G gráf tartalmaz feszítőfát.
\end{framed}
\begin{leftbar}
Ha G-ben van kör, akkor hagyjuk el a kör egy tetszőleges élét, ha a maradék gráfban még mindig van kör, akkor annak is hagyjuk el egy tetszőleges élét, folytatjuk, amíg körmentes lesz a gráf. Ha már nincsen több kör, akkor nézzünk rá, hogy mit kaptunk - az összefüggőség nem sérül, mivel körök egy-egy élét hagytuk el, és pontot se hagytunk el - tehát amit kaptunk, az láthatóan G gráf feszítőfája.
\end{leftbar}
\end{document}